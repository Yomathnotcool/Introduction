\documentclass{article}
\usepackage[utf8]{ctex}

\title{Introduction}
\author{sheldonshelock6 }
\date{May 2021}

\begin{document}

\maketitle

\section{背景介绍}
在数学中,Langlands纲领是一个链接众多学科中的重大的猜想的网络。它链接数论、表示论、几何等等数学领域。这样一个宏伟的数学纲领是由Robert Langlands在1967到1970年之间提出来的,最初Langlands期望找到代数数论中Galois群和自守形式之间的关系,和表示论中局部域和Adele之间的关系。Langlands纲领可以被看作现代数学研究中最大的最重要的研究计划,也被Edward Frenkel称为数学的统一场论。


在众多Langlands的猜想中,有着定义在不同的域上的不同的群,而对于这样不同的数学结构对应有着不同版本的Langlands猜想。比如说对于本文章中要详细讨论的$l$群都有着各种各样的定义。在1967年Langlands提出这系列的猜想后,Langlands纲领发展至今已经有了极大的进化。对于Langlands纲领不同版本的叙述中主要涉及一下概念:
\begin{itemize}
    \item 局部域上的退化群表示,比如说Archimedean局部域,$p$-adic局部域,函数域;\item 全局域上的退化群的自守形式,同样上可以定义在类域上和函数域上;
    \item 有限域,Langlands在他的猜想中并没有涉及这种情况,但是其中有些类似情况;
    \item 更多一般的数域,比如在几何Langlands中研究的复数函数域
\end{itemize}

而在$P$-adic表示的发展之初,最重要的发展是Jacquet和Langlands在1970的书“Automorphic Forms on $GL(2)$”。cite在这本书中,Langlands和Jacquet提出了$P$-adic表示论和数论之间惊人的联系:被称之为不可交换的互反律。这本书,在众多思想方法上,都决定了未来表示论研究的方向。

在1971 Gel'fand 和 Kazhdan把Jacquet和langlands在其书的技术手段和Kirillov的模型和Whittaker的模型从$\mathbf{GL}(2,F)$推广到了$GL(n,F)$。

在1970-1973年Jacquet和Howe在$\mathbf{GL}(n,F)$的表示上得到了一系列重要结论。他们应用的一系列方法构造了一系列的群。

Harish-Chandra将这个复杂问题的证明简化到了一个命题:"对于一个$N$不变向量空间维数的尖表示不会增大到某个给定的值,这仅由$N$决定,其中$N$是群$G$的开子集。"而通过Howe证明的猜想,Bernshtein将这个推广到了更一般的情况。

而对于本文章中,主要是一些初步的结论和基本的证明。第一部分的内容主要研究了局部紧的零维空间和群(称之为$l$空间和$l$群)。随后构造了$l$空间上分布(使用Bruhat cite的方法)。在这样的空间,可以更加简单的实代数簇。并且引入了$l$层的概念,这个概念在表示论的研究中是非常重要的,我们也将会研究$l$层上的分布。

在第二部分内容中,我们将会初步的给出一些关于$l$群的表示一般性的结论。对于其中的一些证明相对复杂,只能略过。事实上在研究代数表示论比一般情形的表示论要简单很多,而且不需要在表示空间中引入拓扑。对于代数表示论的研究可以称之为不可约容许表示的简化版。

这些内容在进一步研究$P$-adic表示理论时会是非常重要的技术手段。比如在证明不变分布的一般性结论时,$l$结构便起到了重要的作用,其断言:如果一个$l$群和自同构$\sigma$作用在$l$层$\mathcal{F}$,则在一些特定条件下$\mathcal{F}$上的任一$G$不变分布在自同构$\sigma$下不变。通过拓扑的语言,或者通过这样的代数几何的语言,在Gel'fand和Kazhdan的文章解决了很多问题。cite
\end{document}
